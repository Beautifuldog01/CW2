\documentclass[11pt]{article}

\usepackage[utf8]{inputenc}
\usepackage[T1]{fontenc}
\usepackage{graphicx}
\usepackage{caption}
\usepackage{subcaption}
\usepackage{amsmath, amsthm, amssymb}
\usepackage[backend=biber,style=numeric,natbib=true]{biblatex}
\usepackage[margin=1.2in]{geometry}
\usepackage{setspace}

\setstretch{1.15}

\addbibresource{references.bib}

% TODO change the title
\title{Modelling of real-life MIP problems}
\author{Ikunbelievable Group\footnotemark[1]}
\date{}

\begin{document}

\maketitle

\renewcommand{\thefootnote}{\fnsymbol{footnote}}
\footnotetext[1]{%
  The Ikunbelievable Group consists of four people who contributed equally to this work:
  \begin{tabular}[t]{@{}l}
    Xiaotian Ji\footnotemark[2] (\texttt{Xiaotian.Ji20@student.xjtlu.edu.cn}), \\
    Qiuyi Chen\footnotemark[3] (\texttt{Qiuyi.Chen2002@student.xjtlu.edu.cn}), \\
    Liyuan Jin\footnotemark[4] (\texttt{Liyuan.Jin20@student.xjtlu.edu.cn}),   \\
    Qin Chi\footnotemark[5] (\texttt{Chi.Qin20@student.xjtlu.edu.cn}).%
  \end{tabular}%
}
\renewcommand{\thefootnote}{\arabic{footnote}}

\section{Introduction}

Consider the problem

\begin{equation*}
  \frac{d^2 u}{d x^2}+u=0
\end{equation*}

consisting of an ODE and a pair of boundary conditions, where $L > 0$.

\section{Problem description}

\section{MIP Model}
Let $G_1(V_1, E_1)$ and $G_2(V_2, E_2)$ be two given graphs, each with their respective vertex and edge sets. Define a positive edge weight function $w : E_1 \cup E_2 \rightarrow \mathbb{R}^+$. We construct a new graph $G(V, E)$, where $V = V_1 \cup V_2$ and $E = E_1 \cup E_2$. The objective is to find a Hamiltonian cycle in $G$ that minimizes the objective function. In order to achieve this, we introduce binary variables $x_{ij}^k$ for edge selection from either $G_1$ or $G_2$.

Formally, let $x_{ij}^k \in \{0, 1\}$ be a binary variable such that:

\begin{equation*}
x_{ij}^k =
\begin{cases}
1, & \text{if edge } (i, j) \text{ from graph } G_k \text{ is selected,} \\
0, & \text{otherwise.}
\end{cases}
\end{equation*}

The goal is to determine the values of $x_{ij}^k$ that lead to a Hamiltonian cycle with the minimum objective function value in the graph $G$.
 The objective function to minimize can be written as:
\begin{equation*}
  \text{minimize} \, f = \sum_{k=1}^{2} \sum_{(i,j) \in E_k} w_{ij}^k x_{ij}^k
\end{equation*}

Subject to the following \textbf{constraints}:

\begin{enumerate}
  \item Each vertex has a total degree of 2, with one incoming edge (in-degree) and one outgoing edge (out-degree) from either graph $G_1$ or $G_2$.

  \begin{equation*}
    \sum_{k=1}^{2} \sum_{j \in V} x_{ij}^k = 2, \quad \forall i \in V
  \end{equation*}
  \item No subtours are allowed (subtour elimination constraint):

  \begin{equation}
    \sum_{(i,j) \in S \times (V \setminus S)} x_{ij}^k \geq 2, \quad \forall S \subset V, S \neq \emptyset, S \neq V, k \in \{1, 2\}
  \end{equation}
  
  Here, $S$ is a subset of $V$, and $(V \setminus S)$ is the complement of $S$ in $V$. (1) ensures that there are no smaller cycles within the Hamiltonian cycle.
  \item The number of edges selected from $G_1$ and $G_2$ are no more than $N_1$ and $N_2$ respectively.

  \begin{equation*}
    \sum_{(i,j) \in E_1} x_{ij}^1 \leq N_1 \quad \text{and} \quad \sum_{(i,j) \in E_2} x_{ij}^2 \leq N_2,  \quad \forall i,j \in V
  \end{equation*}
\end{enumerate}

\subsection*{Variables and Parameters}
In our case, the variables and parameters are defined as follows:
\begin{itemize}
  \item $G_1(V_1, E_1)$ and $G_2(V_2, E_2)$ store the information of airplane's and train's transportation modes.
  \item The vertices $V_1$ and $V_2$ in each graph represent cities in different countries.
  \item $k$ is an index representing the transportation mode in the given graphs. Specifically, $k = 1$ corresponds to the airplane transportation mode in graph $G_1$, and $k = 2$ corresponds to the train transportation mode in graph $G_2$.
  \item If there is an edge between the vertices, it indicates that travel between the cities is accessible using either mode of transportation. We can represent an edge $(i, j)$ as $(i, j) \in E_k$, where $k \in \{1, 2\}$.
  \item The edges store $cost$ and $time$, which represent the one-way ticket price and travel time, respectively. So the weight function $w$ is defined as:
\begin{equation*}
w_{ij}^k = \alpha \cdot \text{cost}_{ij}^k + \beta \cdot \text{time}_{ij}^k, \quad where \quad \alpha + \beta = 1
\end{equation*}

where $\alpha$ and $\beta$ are the weights of the cost and time attributes respectively. 

\end{itemize}
Therefore, our objective function can be further written as:

\begin{equation*}
\text{minimize}\, f = \sum_{k=1}^{2} \sum_{(i,j) \in E_k} (\alpha \cdot \text{cost}_{ij}^k + \beta \cdot \text{time}_{ij}^k) x_{ij}^k
\end{equation*}

\section{Computational Results}
Different weight functions result in different
consequences\cite{lamport1994latex}.
\section{Conclusion}
\section{Appendix}
\section{References}
\printbibliography[heading=none]
\end{document}